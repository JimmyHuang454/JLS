% !TeX root = ../thuthesis-example.tex

\chapter{Specification V2}
本章描述了 JLS 协议第 2 版的具体内容(不含正确性和安全性验证),另付一个基于 Dart 编程语言的具体实现。

\section{约定}
\begin{itemize}
	\item 数组相加有先后顺序,如无特殊标注,一律以大端字节序存储
	\item 如无特殊标注,本文所指TLS均为TLS版本1.3\cite{rescorla2018transport}
\end{itemize}

\section{背景}

对代理协议内容加密和认证是一项广泛的需求,基于应用层的代理协议如 HTTP、SOCK5 等都不支持内容加密\footnote{虽然有认证,但形同虚设},因此出现了如 ShadowSocket 等加密代理协议,随着 TLS 的广泛应用,为了解决维护难、缺少安全性验证、主动探测等问题,又出现了以 TLS 为基础的协议,如 Trojan 协议。

Trojan 可以说是一个完美的协议,但必须要求购买域名,申请域名证书。在某些极端环境,仅允许特定域名通行,Trojan 失效。但预估 IP 白名单模式\footnote{仅限指定IP通行的模式}很快来临,届时我们只能祈祷。


\section{基本原理}

首先考虑一个简化模型:
现有一个客户端、一个服务端和一个仅双方知道的一个密码PWD;如何保证:\newline
\begin{itemize}
	\item 1. 客户端和服务端之间的数据原文没被篡改
	\item 2. 不能被除了客户端和服务端以外的第三方解密
	\item 3. 客户端(服务端)能够识别接收到的数据是否真的来自服务端(客户端)
	\item 4. 收到的通讯数据是否已经接收过(即防止重放攻击)\newline
\end{itemize}

为了实现第 1、2 、3 点,可以采用 AES-GCM 对称加密算法;AES 保证了第 2 点,GCM 保证了第 1 和第 3 点;

为了实现第 4 点,我们可以在 AES-GCM 的基础上往数据原文中加上一个时间戳,但是这种方式有延迟,比如说 vmess 协议,只要攻击速度够快,90 秒内可以无限让客户端或者服务端重复接收已经接收过的数据。

还有一种方法是让客户端和服务端之间进行一次握手,客户端先生成一个随机数 N1,通过 AES-GCM 加密发送 N1,服务端接收后可以解密出来,再生成一个随机数 N2,同样地加密发送给客户端;

完成握手后,就约定以后发送的数据的秘钥都是N=N1+N2+PWD,而且每发一个数据包都要在原文中附上序号(第一个包、第二包……第 x 个包)。这种会增加一次 RTT,但我们可以稍微修改一下 TLS 协议,把其中的随机数(random)换成我们给出的随机数 N1 和 N2,达到交换随机数的目的。

这种方法没有延迟,两个随机数共同组成本次对话的ID,所以该随机数应该要足够随机才能保证能防止重放攻击(Replay Attacks)。

根据 GCM 的认证功能,如果不知道 PWD,就无法解出密文,所以我们就可以检测出客户端/服务端是否假冒,从而把数据转发到我们想要的任何地址。

但为了避免服务器被探测到,只能支持全程加密的 TLS1.3;同样的,如果服务端检测到不是 TLS1.3,直接转发数据,不作处理。



\section{准备}
Client 和 Server 都必须准备一个仅双方知道的密码 PWD 和一个随机数 IV。PWD和IV长度无限制,但都不应该小于 32 字节并应该足够随机。

Server 端必须准备一个伪装站(Fallback Website)地址,伪装站应该优先选取延迟低的、使用 TLS1.3 的、使用 HTTP2 的、不能是HTTP3、DDOS 防御没那么敏感的。

强烈建议至少每1年更换一次密码和IV,最好每三个月更换一次。

\section{Handshake(握手)}
JLS的握手包跟TLS所规定的数据结构和时序完全一致。

\subsection{Client Hello 包}
功能:向服务器发送建立安全信道的请求

\begin{itemize}
	\item 如果 Client 支持 TLS1.3,那么在 Client Hello 的supported\_version拓展必须包含TLS1.3字段。JLS 是建立在 TLS1.3的基础上,所以 JLS 的 Client Hello 包必须要有此字段

	\item Client Hello 中的random字段,必须是通过算法\ref{list:clientFakeRondom}所生成的FakeRandom,通过代码\ref{list:clientFakeRondom}还会生成一个随机数 N1,N1 必须要足够随机,用于防止已知明文攻击(Known Plaintext Attack)

	\item Client Hello 中的session ID字段,应该是随机的

	\item Client Hello 包必须包含key\_share字段和交换秘钥所用算法。Client 必须自行生成正确的key\_share和所用算法(即supported\_groups)发送给 Server

	\item 开发者可以在不修改或缺少 JLS 必要信息的情况下伪造Client Hello头特征,比如说 chrome浏览器的cipher suits列表等,用于伪造 Client Hello 指纹。需要特别注意的是:如果开发者错误地伪造 Client Hello,也就是JLS能够正常使用,但 TLS 不能使用的情况下,会导致明显的特征,比如说 Client Hello 缺少了cipher suits等必要信息,TLS Server 应该发送 HelloRetryRequest,但JLS 认为这是合法 Client,所以是不会发送 HelloRetryRequest

	\item Client Hello 不应该使用early data或pre\_shared\_key,虽然可以做到 0-RTT,但是会导致重放攻击等安全问题

	\item 使用padding extension 是为了希望避免发送长度为256-511字节的Client hello\cite{langley2015rfc},因为有不少TLS server拒绝接收长度小于512字节的Client hello,因此开发者在伪装Client hello 时,应该尽可能使用长度(不含server\_name扩展)大于512字节的Client hello
\end{itemize}

\lstinputlisting[caption=生成Client Hello,label={list:clientFakeRondom}] {data/code/build_client_hello.java}


\subsection{Server 验证}
功能:验证收到的握手是否来自有效的Client
\begin{itemize}
	\item Server 必须通过算法\ref{list:serverCheckClientHello}判断是否为有效 Client

	\item 如果不是有效 Client,直接把接收到的所有数据转发到伪装站,不作其他处理

	\item Server FakeRandom的后8字节不能是\[ 44,4F,57,4E,47,52,44,01 \]和\[ 44,4F,57,4E,47,52,44,00 \],如果是,应该不断重新生成一个Server FakeRandom ,直到合规后再发送 Server Hello

\end{itemize}
\lstinputlisting[caption=检查是否有效Client,label={list:serverCheckClientHello}] {data/code/server_check_client_hello.java}

\subsection{发送Server Hello和证书}
功能:发送Server Hello,并发送伪装站证书
\begin{itemize}
	\item 首先要确认是否有效Client,然后通过算法\ref{list:buildServerHello}生成Server Hello.

	\item 如果是有效 Client,则可以得出来自Client的随机数N1,然后Server要生成自己的随机数N2和Server FakeRandom

	\item Server 必须要根据Client的key\_share和supported\_groups得出共同秘钥S1,并把S1也作为最终秘钥之一来加密数据,以保证前向安全性

	\item 因为 Server 已经验证了 Client 的有效性,所以 Server 证书可以传输随意内容,但包长度应该要伪装站返回的证书一致,建议开发者在软件初始化时获取伪装站的真实证书包;为了方便,返回差不多长度的 Server 证书包也是被允许的,因为这不会影响安全性,但可能成为特征;Client 无需验证该证书是否有效

	\item Session ID应与Client Hello中的一致
	\item support\_group 应只使用 x25519
\end{itemize}
\lstinputlisting[caption=生成Server Hello,label={list:buildServerHello}] {data/code/build_server_hello.java}

\subsection{Client 验证}
功能:验证是否有效Server
\begin{itemize}
	\item 根据算法\ref{list:checkServer}得出验证是否有效Server,并且得出共同秘钥S1。如果不是有效 Server,则 Client 完全按照 TLS1.3 流程处理,即验证证书,并协商出 TLS 最终随机数(秘钥),最后发送合规的 http 请求即可

	\item 如果是有效 Server,则无需验证来自 Server 的证书。
\end{itemize}
\lstinputlisting[caption=验证 server,label={list:checkServer}] {data/code/client_check_server.java}

\section{Application Data}
使用经过x25519算法得出共享密钥S1,最后得出的秘钥finalPWD = PWD + S1,通过AES\_GCM\_256加密发送数据。包结构与TLS一致。
Server和Client都必须各自维护一个自增ID,用于记录已接收和已发送包数量,按照算法\ref{list:encryptData}得出实际发送数据。目前GCM的MAC的长度为默认的16字节,每一个 Application Data 都必须在密文前加上MAC。如果Client或Server收到验证失败的Application data,必要按照正常TLS流程处理。
\lstinputlisting[caption=加密数据,label={list:encryptData}] {data/code/data_encrypt.java}

\lstinputlisting[caption=解密数据,label={list:decryptData}] {data/code/data_decrypt.java}

\section{Dart编程语言实现}
RRS 是一个Trojan和JLS的具体实现。详见地址:

https://github.com/JimmyHuang454/RRS/tree/master/lib/transport/jls

\section{节点分享格式}
示例:
\lstinputlisting[caption=分享格式,label={list:share}] {data/code/share.json}

