% !TeX root = ../thuthesis-example.tex
\chapter{Specification V3}
第3个版本跟TLS完全一模一样,除了random字段是特殊生成的。

\section{基本原理}
因为TLS1.3会生成KeyShare来保证此次对话是唯一的。因为random生成时加入了KeyShare,所以攻击者是无法伪装成Client或Server。

\section{random生成算法}
生成一个随机数N(16字节),通过用户输入的userIV和userPWD,使用AES\_256\_GCM对N加密。加密时,先把ClientHello和ServerHello中的random字段的32字节全部置换为0,得出Hello。

实际使用的:
\begin{itemize}
	\item 原文plainText=N
	\item 随机数iv=sha256(utf8.encode(userIV) + Hello)
	\item 密码pwd=sha256(utf8.encode(userPWD) + Hello)
	\item 附加消息additionalText=null
\end{itemize}

加密后生成出默认长度为16字节的消息认证码MAC和16字节的密文cipherText,通过拼接 cipherText + MAC 得出32字节的 FakeRandom。

\section{PSK处理}
如果Client Hello包含Pre\_Shared\_Key 拓展,生成FakeRandom 时,Pre\_Shared\_Key应该包含真实的Identity参数,但Binder参数应该全置为0(长度应与真实长度一致);生成完FakeRandom并填充进Client Hello后,再计算并填入真实Binder,因为Binder会验证实际使用的Client Hello。

\section{Server 验证}
Server 根据收到Client Hello的random来判断是否有效Client,如果不是有效Client,直接转发到伪装站。如果是有效Client,则不需转发到伪装站,使用自签证书,完整按照TLS的流程处理。

\section{Client 验证}
Client 根据收到的 Server Hello的random来判断是否有效Server,如果不是有效Server,则应表现成一个正常HTTP请求。如果是有效Server,则不需要验证证书是否有效(直接信任),最后按照正常TLS流程处理即可。

\section{安全建议}
\begin{itemize}
	\item 如果开启了0-RTT功能,那么可能遭遇重放攻击
	\item 如果选取支持0-RTT的伪装网站,相应的,你应该开启0-RTT功能来避免特征,而且如果伪装网站支持Session共享(即不同机器不同IP也接受PSK),那么可能会暴露特征
\end{itemize}
