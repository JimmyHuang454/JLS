% !TeX root = ../thuthesis-example.tex

\chapter{常见问题}
回答用户可能经常遇到的问题。

\section{JLS安全性如何?}
JLS充分考虑了各种可能的攻击,如重放攻击,无论是对Client或Server重放,都可以识别出来。身份认证采用带有随机数的AES,而且认证加密内容也是随机的,用于防止已知明文攻击,攻击者必须得出密码和随机数才能正确解密,因此密码长度为32+64字节。即使身份认证被破解了,但还使用ECDH算法,保证了每次对话的密钥都是唯一的。特别注意的是:不应该采用HMAC作为身份认证,因为已知明文的HMAC是非常不安全的。

\section{JLS性能如何?}
有些用户可能会有这样一个想法:代理的数据本来就已经过一次TLS加密,再加一次AES加密是多此一举,因此可以省略一次加密,以提高加密速度。我们认为这种想法是没必要的,实际上一台非常普通的机器,其AES加密性能至少可达1GB每秒的速度,已满足99\%的用户需求,如果你还疑惑,可用此工具来测试你本地的性能,如果你的网络带宽远不及加密速度,那么就无需担忧AES加密带来的性能问题。
